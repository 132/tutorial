% Typography
\usepackage[utf8]{luainputenc}
\usepackage{amsmath}
\usepackage{mathspec}
\usepackage{unixode}
\usepackage{amssymb}
\usepackage{fontspec}
\defaultfontfeatures{Ligatures=TeX}
%% \setmainfont{Adobe Caslon Pro}
%% \setsansfont{Myriad Pro}
%% \setmonofont[Scale=MatchLowercase]{Menlo}
\usepackage{microtype}
\usepackage{setspace}
\usepackage{etoolbox}
\usepackage{csquotes}
\usepackage{ulem}
\usepackage{xstring}
\usepackage{gitinfo}

% Minted
\usepackage{minted}
\renewcommand{\MintedPygmentize}{./pygments-main/pygmentize}
\setminted{encoding=utf-8}
%\setmonofont{freemono}
%\usepackage{etoolbox}
%\AtBeginEnvironment{minted}{\singlespacing%
%    \fontsize{10}{10}\selectfont}
\def\args{linenos,frame=single,fontsize=\normalsize,style=bw}
\newcommand{\makenewmintedfiles}[1]{%
  \newmintedfile[inputlatex]{latex}{#1}%
  \newmintedfile[inputc]{c}{#1}%
}
\expandafter\makenewmintedfiles\expandafter{\args}

% Colors, links, syntax highlighting
\usepackage[dvipsnames,table]{xcolor}
\hypersetup{
    pdftitle={Theorem Proving in Lean},
    pdfauthor={Jeremy Avigad, Leonardo de Moura, Soonho Kong},
    colorlinks,
    linkcolor=Sepia,
    citecolor=Periwinkle
}

% warning box
\usepackage{mdframed}
\definecolor{warningborder}{HTML}{FAEBCC}
\definecolor{warning}{HTML}{FCF8E3}
\newenvironment{warning}{%
  \begin{mdframed}[backgroundcolor=warning,linecolor=warningborder]%
}{\end{mdframed}}

\usepackage[acronym,toc]{glossaries}
\include{Glossary}
\makeglossaries
\makeindex

% Localization
\usepackage{polyglossia}
\setdefaultlanguage{english}

% Mathematical symbols
%% \usepackage{savesym}
%% \savesymbol{iint}
%% \savesymbol{iiint}
\newcommand\xor{\oplus}

% Advanced command definitions
\usepackage{xparse}

% Figures and subfigures
\usepackage{float}
\usepackage{rotating}
%\usepackage{subcaption}
\usepackage{array}
\usepackage{calc}
\usepackage{framed}
\usepackage{wrapfig}
\usepackage{longtable}
\newcolumntype{C}[1]{>{\centering\arraybackslash$}p{#1}<{$}}
\usepackage[export]{adjustbox}
% Fancy figure command
\DeclareDocumentCommand{\illustration}{
   m     % sub-path to illustration, /Illustrations/{this}.pdf, mandatory
   O{.8} % width as fraction of line width, optional
   o     % caption, optional
   o     % label, optional
  }{
  \begin{figure}[ht!]
    \centering
    \includegraphics[width=#2\linewidth]{./Illustrations/#1.pdf}
    \IfValueTF{#3}{\caption{#3}}{}
    \IfValueTF{#4}{\label{#4}}{}
  \end{figure}
}

% Exercise Environment
\newtheorem{exercise}{Exercise}[chapter]
\newenvironment{exercise*}
  {\renewcommand\theex{\thechapter.\arabic{ex}\rlap{$^*$}}\ex}
  {\endex}

% Special subsection commands
\definecolor{shadecolor}{HTML}{AABAD1}
\DeclareDocumentCommand{\advanced}{
   o     % extra caption, optional
  }{
  \begin{shaded}
      \begin{wrapfigure}{l}{.2\linewidth}
          \vspace{-8pt}
          \includegraphics[width=\linewidth]{./Illustrations/Propeller/Propeller.pdf}
      \end{wrapfigure}
      \noindent This is an optional, in-depth section. It almost
      certainly won't help you write better software, so feel free to
      skip it. It is only here to satisfy your inner geek's curiosity.
      \IfValueTF{#1}{#1}{}
  \end{shaded}
}
% Chapter markup
\usepackage{tikz, blindtext}
\makechapterstyle{box}{
  \renewcommand*{\printchaptername}{}
  \renewcommand*{\printchapternum}{
    \flushright
    \begin{tikzpicture}
      \draw[fill,color=black] (0,0) rectangle (2cm,2cm);
      \draw[color=white] (1cm,1cm) node { \chapnumfont\thechapter };
    \end{tikzpicture}
  }
  \renewcommand*{\printchaptertitle}[1]{\flushright\chaptitlefont##1}
}
\chapterstyle{box}
% Title page markup
\usepackage{geometry}
\usepackage{titlesec}
\usepackage{datetime}
\makeatletter
\newlength\drop
\renewcommand{\maketitle}{
  \thispagestyle{empty}
  \begingroup
  \drop = 0.1\textheight
  \vspace*{\baselineskip}
  \vfill
  \hbox{%
    \hspace*{0.22\textwidth}%
    \rule{1pt}{\dimexpr\textheight-28pt\relax}%
    \hspace*{0.05\textwidth}%
    \parbox[b]{0.73\textwidth}{%
      \vbox{%
        \vspace{\drop}
               {\Huge\bfseries\raggedright\@title\par}\vskip2.37\baselineskip
               {\raggedleft
                 \Large\bfseries
                 \expandarg{\StrSubstitute{\@author}{,}{\par}}
                 \par}
               \vspace{0.5\textheight}
               {\bfseries
                 \href{https://github.com/leanprover/tutorial/commit/\gitHash}{Version \gitAbbrevHash},
                 updated at \gitAuthorIsoDate}
      }% end of vbox
    }% end of parbox
  }% end of hbox
  \vfill
  \null
  \endgroup
  \thispagestyle{empty}
  \par
  \noindent
  Copyright (c) 2014--2015, Jeremy Avigad, Leonardo de Moura, and Soonho Kong. All rights reserved.
  Released under Apache 2.0 license as described in the file LICENSE.

 \clearpage
 % \thispagestyle{plain}
 % \par
 % \vspace*{.3\textheight}{
 %   \centering
 %   \emph{To ...}
 %   \par
 %   \clearpage
 % }
}
\makeatother
